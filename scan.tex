% \textsf{Vulnerability scanning using metadata.}
The purpose of the Thing Descriptions is to enable easier discovery and use of 
devices.  However, the flip side of this is that it may also become easier
for attackers to discover vulnerable devices, or to infer private information
simply from the type of devices available.

The first line of defense is to protect access to discovery services such
as Thing Directories.  If Thing Directories can only be accessed by authorized
users, a number of types of attacks become more difficult.  Since a Thing
Directory is simply a web service, it can be protected with normal web service
authorization mechanisms.  However, a given system may provide other means of 
discovery, such as broadcast responses or extended DNS entries.  These may not
be as easy to protect.  Note that in order to support a protected Thing Directory
a protected onboarding process is needed to associate devices with a given
Thing Directory and of course to provide authorized users with appropriate
credentials to access the Thing Directory.

Assuming an attacker can access Thing Descriptions, however, they may be
able to exploit them in various ways.  First, they could try to analyze the 
interactions available themselves for known vulnerabilities.  While the 
Thing Descriptions intentionally omit information about the software stack
providing the service, an attacker may be able to use fingerprinting to
associate a particular Thing Description to a particular device with known
vulnerabilities.  The vulnerabilities may not even be over the network;
for instance, an attacker may search for ``smart locks'' that can be defeated
with a known physical attack.
The flip side of this, however, is that a System Maintainer can apply the same
tools to scan for devices with vulnerabilities, in order to identify devices
at risk so that mitigations can be put in place (such as scheduling updates to
a device's firmware).
Unlike a malicious attacker, the System Maintainer also has the advantage that
they can legitimately access \emph{all} Thing Descriptions in a system.

Even if an attacker cannot determine vulnerabilities, they may still be able
to determine personal information about a user by the kinds of devices they
have installed.  For example, knowing that someone has a baby monitor lets you
infer that they probably have a young child.
Semantic tags make this information explicit but even without tagging,
fingerprinting may be able to associate a set of interactions with
a specific class of devices.  The mitigation of this kind of attack is to
protect the Thing Descriptions themselves and make them available
\emph{only} to authorized users.
