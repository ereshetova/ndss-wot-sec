%\textsf{A basic intro to to the WoT threat model \cite{Wot2017sec}.}

Due to the large diversity of devices, use cases, deployment scenarios and various requirements for WoT Things, it is impossible to define a single WoT threat model that would fit every case.
Instead we have created an overall WoT threat model~\cite{Wot2017sec} that can be adjusted by OEMs or other WoT system users or providers based on their concrete security requirements.

%The WoT threat model defines the following key WoT Assets that are important from security point of view:	\textit{Thing Description}, \textit{Thing User Data} (any user data transferred via WoT network, such as video streams, sensor data etc.), \textit{Thing Provider Data} (WoT application scripts and their configuration data), \textit{WoT Basic Security Metadata} (all provisioned security medatada), \textit{WoT Controlled Environment} (physical environment that can be affected by WoT Things), \textit{WoT Thing and Infrastructure Resources} (resources of devices providing WoT Things and overall WoT network) and \textit{WoT Behavior Metrics} (all indirectly transmitted information).
%Some of these assets might be absent and/or have different trust model (i.e who should have a legitimate access and to what extent) depending on deployment scenario. 

The threat model defines security-relevant WoT assets and a set of WoT threats on these assets that can be in- or out-of-scope based deployment scenario, security objectives, risks etc.
For example, in a smart home scenario involving WoT things that record audio/video information a privacy aspect is very important and therefore this type of data has high confidentiality requirements.
On another hand, in industry automation scenario involving WoT things that control some safety-critical infrastructure, confidentiality might not be the highest priority, but availability of the WoT Thing and Infrastructure Resources and protection of WoT Controlled Environment is of topmost importance.

Being a distributed system brings an additional complexity angle to the WoT Threat Model and the choice of relevant security mitigations.
One cannot anymore rely on standard communication infrastructure and protocols (like HTTPS) to provide an end-to-end security between all communicating parties, and instead WoT needs to enable supporting different set of security mechanisms (potentially nested or interdependent) that can be combined into single end-to-end security solution.  
  
 

