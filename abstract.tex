The W3C Web of Things (WoT) WG has been developing an interoperability standard for IoT devices that includes as
its main deliverable a ``Thing Description'': a standardized representation the metadata of an 
IoT device, including in particular its network interface, but also allowing for semantic annotation.
Relative to other approaches to IoT, such metadata has at least four major implications.
First, it allows for system-wide vulnerability analysis, 
which can be both a risk and an opportunity.
Second, metadata can enable end-to-end security in multistandards networks,
avoiding exposing data within bridges otherwise needed for connecting standards pairwise.
Third, metadata supports service and device discovery,
which raises the question of how to limit discovery to authorized agents.
Fourth, metadata can enable distributed security mechanisms for access control and micropayments.
To the extent that metadata access can be decentralized, decentralized mechanisms for security can
be supported, although several practical issues currently make this difficult to fully support.
