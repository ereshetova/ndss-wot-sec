Many issues arise when trying to access IoT devices
that are located on a ``local'' network,
for example behind a NAT in a Smart Home use case.
In addition, devices may be accessible over multiple
routes and protocols.

When one hears of the term ``Web of Things'' the
assumption is that HTTP (or ideally, HTTPS) will be used to
access IoT devices. This may or may not be the best choice
for a variety of reasons.  However, use of HTTP
allows devices acting as HTTP servers to be accessed directly
from web browsers and provide their own user
interfaces via HTML.  To provide better security, one would
naturally want to use HTTPS rather than HTTP.
Unfortunately HTTPS and the certificate system behind it has
been designed for globally accessible web sites, not devices
behind NATs that may or may not have a globally accessible
address and may or may not even be connected to the internet.
In particular, certificate revocation checks will not work if
the network both devices are on is not connected to the internet
(for example, if an adhoc network is used, with the device acting as
an access point) and certificates will not generally be able 
to tie the identity of a device to a particular URL. 

Both of these problems can be avoided by using a cloud proxy
or mirror (digital twin) for the device.  In that case a server
in the cloud is used as an intermediary.  Unfortunately,
this requires an active internet connection even to use 
local devices, and is also relatively high latency and
bandwidth-inefficient.
Various mechanisms have been explored to deal with this issue\cite{httpslocal2017}
but no generally accepted mechanism has been adopted.
Therefore a metadata standard needs to be able to deal with a variety
of approaches.

Other secure protocols, like CoAPS, 
do not have the same assumptions as HTTPS and can be used
more easily in an segmented network.  
The WoT, despite its name, can also work with these standards.

However, even with CoAP, there is another issue: the URLs
used to access the same device can vary depending on whether the
device is accessible on the local network or should be accessed
via a global URL (either via NAT port forwarding, a proxy, or
a digital twin).  A Thing Description can include multiple links
for each interaction, so in theory both local and global links
can be included in a single Thing Description.
However, a user of a Thing has no easy way to tell if it is on
the same local network as another target Thing,
and if not, the ``local'' links
won't work... or will connect to a different device.  Another
approach would be for the Thing Directory to return a modified
Thing Description with local links to clients that it knows 
are on the same local network.
The first approach, multiple
links, has the problem of broken links (and possibly even 
connecting to the wrong device, if some other mechanism is not
used to control access).
The second approach raises the possibility
of a malicious Thing Directory, or a network attacker spoofing a
Thing Directory, to redirect clients to fake devices.
