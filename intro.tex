The economic impact of the IoT is strongly influenced by how well devices from 
different manufacturers can interoperate.
Very often interoperability is taken for granted when estimating the business
benefit of IoT. 
However, a recent study \cite{McKinley2015}
concluded that 40\% to 60\% of the 
benefit of IoT will be unattainable if devices do \emph{not} interoperable,
due to use cases that cannot be satisfied by a single manufacturer.

Unfortunately full interoperability is hard to acheive.
There are currently many competing IoT standards under development.
Most of these standards are prescriptive.
In a prescriptive standard, devices are validated against specific requirements
and typically all validated devices will interoperate.
In addition, it is possible to bridge multiple standards so that
devices validated against one prescriptive standard can communicate with
devices using another standard by translating communication protocols and payloads.

However, the prescriptive approach has some weaknesses.
In particular, there are always going to be devices that follow older standards.
There are decades-old devices in particular domains, such as building and factory
automation, that are now being connected to the IoT.
These devices represent major investments and cannot be economically replaced with newer,
standards-conforming devices.  This is the ``brownfield'' problem.
In addition, today devices are also being deployed that have not been validated
against any particular standard, but may use common technologies such as JSON and HTTP.

As an alternative, the W3C Web of Things WG has been developing a \emph{descriptive} 
approach to IoT interoperability.
In this approach,
metadata is provided that describes how to communicate with a 
particular device.
The metadata itself is standardized but flexible enough to describe a wide variety of
IoT network interfaces.
With this approach, devices can but do not have to be prevalidated against 
a particular standard before being deployed.
They can be described after the fact, solving the brownfield problem and allowing
older devices to and devices satisfying different standards to be integrated into
a unified system.  

This approach has both risks and opportunities from a security point of view.

Most obviously, devices may vary widely in their support for security,
so the system needs to manage different levels of trust for different devices.
Devices from different manufacturers may also take different approaches to
security and this make cause integration challenges, even if the necessary
information is provided in the metadata.

Beyond this basic concern, pervasive metadata raises several other issues
from a security perspective.  In this paper we discuss four major issues:
\begin{itemize}
\item \textbf{Vulnerability scanning.}
Providing information about what devices can do makes it easier to scan for devices with 
vulnerabilities.  It may also be possible to plan attacks that take advantage of vulnerabilities
in multiple devices. 
However, this risk can in fact be an opportunity as scanning for devices with
vulnerabilities is necessary to identify devices whose vulnerabilities need to be mitigated.
\item \textbf{End-to-end security.}
Metadata enables end-to-end security in multistandards networks.
If metadata is used to push payload adaptation to endpoints then 
communication payloads can be encrypted end-to-end.  This contrasts
with systems that use local bridging between multiple IoT standards
which requires opening (and usually deencrypting) data in potentially-vulnerable gateways.
\item \textbf{Secure discovery.}
Information about how to use a service, and ideally even its existence, should not
be disclosed to agents without the authorization to use it.
The WoT approach allows powerful semantic searches to be used for discovery.
How can this capability be made available while still securing the metadata?
\item \textbf{Security Mechanism Enabling.}
Metadata may be provided to enable specific security mechanisms,
as well as features with security implications such as payment or scripting.
What mechanisms are needed and what data needs to be provided?
\end{itemize}

The next few sections first introduce the W3C Web of Things draft standard,
focusing on the Thing Description metadata format.  
Then the security model for the WoT will be introduced.
This includes a model of stakeholders, assets, attackers, and threats.
Once this context has been established, we will discuss in detail these
four issues.

