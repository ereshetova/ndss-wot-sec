The economic impact of the IoT will be strongly dependent on how well devices from 
different manufacturers can interoperate.
Very often interoperability is taken for granted when estimating the business
benefit of IoT. 
However, if devices do \emph{not} interoperate,
a recent study \cite{McK2015a} concluded that 40\% to 60\% of the 
benefit of IoT will be unattainable,
due to the inability to address use cases that cannot be satisfied by a single manufacturer.

Unfortunately full interoperability is hard to achieve.
There are currently many competing IoT standards under development,
each of which is attempting to address this problem.
Most of these standards are prescriptive.
In a prescriptive standard,
devices are validated against specific requirements.
Typically the goal is that devices validated against particular standard will interoperate with
other devices also validated against that standard.
In addition, it is possible to bridge multiple standards so that
devices validated against one standard can communicate with
devices using another standard by translating communication protocols and payloads.
Of course if one standard comes to dominate bridging will be unnecessary.
However, so far such unification seems to be elusive and may be impossible due to
divergent requirements in different but overlapping IoT subdomains.

Unfortunately the prescriptive approach has some weaknesses.
In particular, there are always going to be devices that follow older standards.
There are decades-old devices in particular domains, such as building and factory
automation, that are just now being connected to the IoT.
These devices often represent major investments and cannot be economically replaced with newer,
standards-conforming devices.
This is the ``brownfield'' problem.
In addition,
new devices are being deployed today that have not been validated
against any particular IoT standard,
even if they use other standards such as JSON and HTTP.

As an alternative to the prescriptive approach,
the W3C Web of Things (WoT) Working Group has been developing a \emph{descriptive} 
approach to IoT interoperability.
In the descriptive approach,
metadata is provided that describes how to communicate with each particular device.
The metadata itself is standardized but flexible enough to describe a wide variety of
IoT network interfaces.
With this approach,
devices can but do not have to be prevalidated against 
a particular standard before being deployed.
They can be described after the fact,
and do not need any modification to be
used as part of a Web of Things system.
This solves the brownfield problem and allows
older devices as well as devices satisfying different IoT 
standards to be integrated into a unified system.  

This approach has both risks and opportunities from a security point of view.

Most obviously, IoT devices, even those conforming to a prescriptive IoT standard,
may vary widely in their support for security.
Therefore a Web of Things system
needs to manage different levels of trust for different devices.
Devices from different ecosystems or manufacturers may also take different approaches to
security and may use different security mechanisms. 
This may cause integration challenges, even if the necessary
information is provided in the metadata.

Beyond this basic concern,
the availability of pervasive metadata raises several other issues
from a security perspective.
In this paper we discuss four major issues:
% NOTE: want to use {description} here but it seems to be broken...
\begin{enumerate}
\item \textbf{Vulnerability analysis:}
Providing information about what devices can do makes it easier to 
automatically scan for devices with vulnerabilities.
An attacker may also use this information to plan attacks that take advantage of 
vulnerabilities in multiple devices. 
However, for the system manager, scanning can also be an opportunity 
to identify devices whose vulnerabilities need to be mitigated.
\item \textbf{End-to-end security:}
Metadata enables end-to-end security in networks of IoT devices using multiple standards.
If metadata is used to push payload adaptation to endpoints then 
communication payloads can be encrypted end-to-end.  
This contrasts with systems that use local bridging to connect devices from multiple IoT standards.
Local bridges require opening (and usually re-encrypting) data in potentially-vulnerable gateways.
\item \textbf{Secure discovery:}
Information about how to use a service, 
and ideally even its existence, should not
be disclosed to agents without the authorization to use it.
The WoT approach allows powerful semantic searches to be used for discovery.
How can this capability be made available while still securing the metadata?
\item \textbf{Security mechanism enabling:}
Metadata may be provided to enable specific security mechanisms,
as well as features with security implications such as payment or scripting.
What mechanisms are needed and what data needs to be provided?
Also, depending on how the metadata is made available, it may or may not be 
possible to support decentralized approaches to security.
\end{enumerate}

The next few sections first introduce the W3C Web of Things draft standard,
focusing on the Thing Description metadata format.  
Then the security model for the WoT will be introduced,
which includes a model of stakeholders, assets, attackers, and threats.
Once this context has been established, we will discuss in detail the above
four issues.

